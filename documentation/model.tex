\documentclass[12pt]{article}
\usepackage{geometry}                % See geometry.pdf to learn the layout options. There are lots.
\geometry{letterpaper}                   % ... or a4paper or a5paper or ... 
%\geometry{landscape}                % Activate for for rotated page geometry
\usepackage[parfill]{parskip}    % Activate to begin paragraphs with an empty line rather than an indent
\usepackage{graphicx}
\usepackage{amssymb}
\usepackage{epstopdf}
\DeclareGraphicsRule{.tif}{png}{.png}{`convert #1 `dirname #1`/`basename #1 .tif`.png}

\newcommand{\Ro}{\mathcal{R}_0}

\title{Latent model}
\author{Dushoff, Li, Champredon}
%\date{}                                           % Activate to display a given date or no date

\begin{document}
\maketitle



\section{Background}

Given an observed incidence curve, we want to estimate epidemiological parameters and forecast future incidence.

\section{Epidemiological Model}

The model is based on the renewal equation, with heterogeneity and intervention added.

Incidence at time $I_t$ is supposed to be defined as:

\begin{equation}
I_t = S_t^{1+\alpha}\, \Ro \left( \sum_{j=1}^{\ell} e^{-A_t} g(j) I_{t-j}   + \epsilon \right)
\label{eq:renewal}
\end{equation}
with $\alpha\geq 0$ a parameter modelling contact heterogeneity, $A_t$ representing all interventions effectiveness at time $t$, $S_t$ the proportion of the effective population composed of susceptible individuals, $g$ the generation interval distribution and parameter $\ell$ the maximum value of the generation interval. Parameter $\epsilon$ is a small quantity that helps with numerical convergence of the MCMC.

It is assumed that 
$$ g(t) = \frac{1}{G}\, t^{g_1 -1} e^{-\frac{t}{\ell\, g_2}}$$
with $g_1$ and $g_2$ two shape parameters and $G=\int g$ the normalizing constant.

The depletion of susceptibles is taken into account simply by subtracting incidence at each time step:
\begin{equation}
S_{t+1} = S_t - I_t + \epsilon
\end{equation}



An observation process is modelled with a Poisson distribution.


NOTE DC: there is something I don't understand with \texttt{preInc}:
$$I_k = \epsilon + S_k \rho (1-(1+\frac{preInc_k}{S\rho\kappa})^{-\kappa})$$
where $\rho$ is the mean reporting rate

\end{document}  